\chapter{Conclusion\label{cha:chapter7}}

 \begin{figure}[h]
 	\centering
 	\includegraphics[scale=0.6]{EPS!.png}
 	\caption{EPS assembly (PDU, PPU, Battery board}
 	\label{fig: EPS!}
 \end{figure}

This thesis described the development of the Power Distribution Unit that is designed in combination with Power Processing Unit and Battery case Unit to supply a 6U satellite with  continuous power.\\  


Massive work was made to accomplish all requirements and develop the PDU.

 One of the biggest challenges during the PDU design was the switch selection, and their placement on the PCB layout, taking into account the amount of components on the layout and the power channels requirements such as needed voltage, current, the trace width, size and ability to determine an over-current. Another big challenge was the layout structure of the PDU. Due the fact that the PDU should not only be the distribution board for one Descartes Mission, but an universal distribution unit. PDU might be adjusted for different upcoming missions without change of the design, but making little patch such as resistor replacement made master thesis arduous and interesting at the same time. During the tests PDU had some issues with a voltage drop on the Hispico line while initiation of S-band transceiver. The voltage drop occurred due to the high inrush current of the Hispico transceiver.    However, after adding an extra capacitors on the output of the switch, voltage drop was reduced and results were satisfactory.    \\

PDU was tested using microcontroller Nucleo STM32L073Rz with PSU as a power source and with an electrical load to simulate the load. 

After successful results, the PDU was tested using flight hardware such as PPU, which included a microcontroller to control the switches and receive data from current sensors, a battery board that provided power for the PDU, and payloads. 

Power Distribution Unit fits comfortably withing the mechanical structure of the 6U satellite, all the connectors are easily accessible. 
\\
The Power Distribution Board was designed by using Altium Designer and fabricated by Würth Electronics GmbH. All components were soldered by hands and tested according to the test procedure, described in the Chapter \ref{6}. Most of the used components have flight heritage from previous missions, which makes the PDU more promising for the upcoming projects.  \\ 
PDU was developed following the standardized design which includes the board dimensions (90.16x95.9mm) and PC104 connector location. In addition, all data signals are connected to th PC104 connector through the zero ohm resistors, which makes the PDU much more flexible in case of using the PDU in other BUS configuration. 
Due to the upper mention standardized design of the PDU, it is allows PDU to be implemented for a 1U, 2U and 3U satellites. \\
Although PDU was developed and successfully tested for a Descartes Mission, there is still space for a future contributions and improvements. The first possible improvement would be a placement of a zero ohm resistor in the parallel to the isolator. This improvement will give an opportunity to choose if isolator is needed. The reason for this improvement occurred while the temperature sensors test connected to the K11. Initially K11 connector was designed for a payload. According to the functional requirements FR4, PDU shall have isolation for every payload node. However, for a Descartes mission K11 connector was used for a thermal sensors which are part of the satellite bus and have their own isolation. Although TPS203x switches have a flight heritage, relatively simple and robust, they are bulky and can not be adjusted for a current limit by changing resistance value, which is a sensitive disadvantage which might be changed in the future versions.  \\ \\

In conclusion, Power Distribution Board was successfully designed,  manufactured, tested according to all requirements and ready to fly. 
