\chapter{Fundamentals and Related Work\label{cha:chapter2}}


Nowadays most of the companies designing their Electrical Power System boards for a nano cubesats as whole unit, consisted of all important devices for power management. This configuration is common and allowed to create space for a hardware in the tiny cubesat which is essential for a nano satellites due to its limited size and mass. 
\\
\\ Despite the fact that mechanical space of the satellite is important, the space on the EPS board is also limited for a components which is limiting satellite possibilities with bigger missions and more developed bus.  

\section{Technologies \label{sec:tech}}

One unit configuration is a common type of an EPS design, which is mostly used by nano cubesat developers. This type of EPS design allowed to combine all components of the EPS in one unit to save mechanical space for the rest of the bus hardware and a payload of a nano cubesat. Second type of technology is a separated type of the EPS which is divided in to Power Processing Unit (PPU) which is responsible for a power generation from a solar panels, battery charging and balancing as well as power processing and power convertation and Power Distribution Unit (PDU) which has a function of the power distribution. PDU consist mostly of switchers and current sensors, this architecture allowed to place significant amount of switchers and connectors for a payloads, which is necessary for a missions requiring amount of payload connections which will not be enough for a standard one unit EPS board type. 

\subsection{Technology A\label{sec:aaa}}

It's always a good idea to explain a technology or a system with a citation of a prominent source, such as a widely accepted technical book or a famous person or organization. 

Exmple: Tim-Berners-Lee describes the ''WorldWideWeb'' as follows:
\\
\textit{''The WorldWideWeb (W3) is a wide-area hypermedia information retrieval initiative aiming to give universal access to a large universe of documents.''} \cite{timwww}
\\
\\
You can also cite different claims about the same term.
\\
According to Bill Gates \textit{''Windows 7 is the best operating system that has ever been released''} \cite{billgates} (no real quote)
In opposite Steve Jobs claims Leopard to be \textit{''the one and only operating system''} \cite{stevejobs}

If the topic you are talking about can be grouped into different categories you can start with a classification.
Example: According to Tim Berners-Lee XYZ can be classified into three different groups, depending on foobar \cite{timwww}:
	\begin{itemize}
		\item Mobile X
				\vspace{-0.1in} 
		\item Fixed X
				\vspace{-0.1in} 
		\item Combined X
 	\end{itemize}

\subsection{Technology B\label{sec:bbb}}

For internal references use the 'ref' tag of LaTeX. Technology B is similar to Technology A as described in section \ref{sec:aaa}.

\newpage

\subsection{Comparison of Technologies\label{sec:comp}}

\begin{table}[htb]
\centering
\begin{tabular}[t]{|l|l|l|l|}
\hline
Name & Vendor & Release Year & Platform \\
\hline
\hline
A & Microsoft & 2000 & Windows \\
\hline
B & Yahoo! & 2003 & Windows, Mac OS \\
\hline
C & Apple & 2005 & Mac OS \\
\hline
D & Google & 2005 & Windows, Linux, Mac OS \\
\hline
\end{tabular}
\caption{Comparison of technologies}
\label{tab:enghistory}
\end{table}

\section{Standardization \label{sec:standard}}

This sections outlines standardization approaches regarding X.

\subsection{Internet Engineering Task Force\label{sec:itu}}

The IETF defines SIP as '...' \cite{rfcsip}

\subsection{International Telecommunication Union\label{sec:itu}}

Lorem Ipsum...

\subsection{3GPP\label{sec:3gpp}}

Lorem Ipsum...

\subsection{Open Mobile Alliance\label{sec:oma}}

Lorem Ipsum...

\section{Concurrent Approaches \label{sec:summ}}

There are lots of people who tried to implement Component X. The most relevant are ...